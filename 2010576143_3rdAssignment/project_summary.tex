\section{Abstract }
Today is the day of science and technology. Many types of modern inventions have made human life easier. One of them is robots. The study of the ways in which humans interact with robots is a robotics field. By the invention of robots human-computer interaction have been possible. Nowadays there are many robotics industries. The robotics industry is mainly focused on the development of conventional technologies. These technologies improve efficiency and reduce the amount of repetitive works sothat humans activities in a specific task could be minimized. To achieve this goal, robotics industries must train their technical employees to invent the robot when performing tasks. Today the study on machine learning and artificial intelligence spread in broad. As a result robots are being updated frequently. Configuration  and  technical  programming  for  creating robots also being advanced. We know that for creating a robot we need to study about hardware, some robotic algorithms and programming languages and natural control over a robot to generate wide acceptance and massive use in the performance. In this paper I present the challenges in the design, implementation and testing of a hand-based interface to control two robotic  arms and the benefits of this technology that is between robotics and human interaction. The demand for automation in various industrial sectors is increasing rapidly, leading to a growing need for versatile and reliable robotic arms. The aim of this project is to design and develop a multi-purpose robotic arm that can be used in a wide range of applications, including material handling, assembly, and inspection.