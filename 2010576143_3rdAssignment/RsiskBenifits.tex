\section{Risks and benefit of robotic arms}
\subsection{Risks of robotic arms}
Technical risks: Robotics arms are complex systems that require specialized knowledge and skill to design, build, and maintain. Technical risks may arise from issues such as software bugs, hardware failures, or human error during operation.\\
Market risks: The market for robotics arms may be highly competitive, and there may be limited demand for the specific type of arm being proposed. Additionally, changes in technology and market conditions may render the arm obsolete before it can be fully developed or sold.\\
Financial risks: The cost of developing a robotics arm can be high, and there may be limited funding available to support the project. Additionally, the cost of manufacturing and selling the arm may exceed the expected revenue, leading to financial losses.
\subsection{Benefits of robotic arms}
Improved efficiency: Robotics arms can perform repetitive tasks more quickly and accurately than human workers, improving productivity and reducing the risk of human error.\\
Increased safety: Robotics arms can perform tasks in hazardous environments, such as in extreme temperatures or hazardous materials, reducing the risk of injury to human workers.\\
Potential for cost savings: The reduced cost of labor and materials associated with using robotics arms may lead to significant cost savings compared to traditional manufacturing methods.\\
Potential for new business opportunities: The development of new robotics arms may create new business opportunities, such as the sale of arms to other companies or the development of new products and services.\\